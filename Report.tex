\documentclass{article}
\usepackage[utf8]{inputenc}
\usepackage[UTF8]{ctex}
\usepackage{float}
\usepackage{amsmath}
\usepackage{amssymb}
\usepackage{booktabs}
\usepackage{gensymb}
\usepackage{geometry}
\usepackage{color}
\usepackage{graphicx}
\usepackage{subcaption}
\usepackage{listings}
\lstset{
    numbers=left,  % 行号标在左边
    frame=ltrb,  % 显示四周边框
    frameround=tttt,  % 设置边框圆角
    keywordstyle=\color{blue}, % 设置关键字颜色为黑色
    backgroundcolor=\color[RGB]{245,245,244}
}
% \usepackage[framed,numbered,autolinebreaks,useliterate]{mcode}


\title{机器人学课程设计}
\author{姓名:朱楷文\\ 学号:520030910178\\ 班级:F2003001 }
\date{}

\geometry{a4paper,left=2.18cm,right=2.18cm,top=2.54cm,bottom=2.54cm}

\begin{document}

\maketitle

\section{机械臂构建}
\subsection{实现思路}
Puma560 机械臂的 D-H 参数如表 \ref{DH} 所示, 其中 $\theta_i,\  i=1,2,\cdots,6$ 为关节变量, $a_2, d_3, a_3, d_4$ 为定值. 本设计中, 各关节变量的取值范围均为 $(-90\degree, 90\degree)$, $a_2, d_3, a_3, d_4$ 分别设为 100, 20, 10, 100. 据此建立各连杆并将其连接, 即可得到机械臂.
\begin{table}[H]
    \centering
    \caption{Puma560 机械臂的 D-H 参数表}
    \label{DH}
    \begin{tabular}{ccccc}
    \toprule
    $i$ & $\alpha_{i-1}$ & $a_{i-1}$ & $d_i$ & $\theta_i$  \\
    \midrule
    1   &  0 & 0 & 0 & $\theta_1$ \\
    2   &  $-90\degree$ & 0 & 0 & $\theta_2$ \\
    3   &  0 & $a_2$ & $d_3$ & $\theta_3$ \\
    4   &  $-90\degree$ & $a_3$ & $d_4$ & $\theta_4$ \\
    5   &  $90\degree$ & 0 & 0 & $\theta_5$ \\
    6   &  $-90\degree$ & 0 & 0 & $\theta_6$ \\
    \bottomrule
    \end{tabular}
\end{table}

\subsection{结果}
构建出的机械臂如图 \ref{robot} 所示.
\begin{figure}[H]
    \centering
    \includegraphics[width=0.4\linewidth]{robot.png}
    \caption{机械臂示意图}
    \label{robot}
\end{figure}

\subsection{代码}
构建机械臂的函数代码如下.
\medskip
\lstinputlisting[language=Matlab]{create_robot.m}


\section{工作空间可视化}
\subsection{实现思路}
机械臂的工作空间是末端执行器可达到的空间范围, 即, 对于工作空间中的一个点 $\mathbf{p}$, 在关节空间中存在一个点 $\mathbf{q}$, 使得 $\mathbf{f}(\mathbf{q}) = \mathbf{p}$, 其中 $\mathbf{f}$ 为正运动学函数. 因此, 对关节空间中的每个点应用正运动学函数, 即可得到工作空间.\par
实际操作中, 为了可视化工作空间, 由于不可能遍历关节空间中的所有点, 我们在其中进行均匀的随机采样, 对采样得到的每个点应用正运动学函数, 绘制出得到的位置. 若样本足够代表整个关节空间, 即可实现工作空间的可视化. 本设计选择样本容量为 20000.


\subsection{结果}
构建机械臂, 采样并计算后得到的工作空间如图 \ref{workspace} 所示, 其中每个红点表示采样并计算后得到的工作空间中的一个点. 工作空间在 X-Y 平面, Y-Z 平面, Z-X 平面上的投影如图 \ref{projection} 所示.
\medskip
\begin{figure}[H]
    \centering
    \includegraphics[width=0.63\linewidth]{task1.png}
    \caption{机械臂的工作空间}
    \label{workspace}
\end{figure}
\medskip
\medskip
\begin{figure}[H]
    \centering
    \subfloat[\label{xy}X-Y 平面上的投影]{\includegraphics[width=0.3\linewidth]{task1xy.png}}
    \hspace{3pt}
    \subfloat[\label{yz}Y-Z 平面上的投影]{\includegraphics[width=0.3\linewidth]{task1yz.png}}
    \hspace{3pt}
    \subfloat[\label{zx}Z-X 平面上的投影]{\includegraphics[width=0.3\linewidth]{task1zx.png}}
    \caption{\label{projection}工作空间在各平面上的投影}
\end{figure}
\medskip

\subsection{代码}
可视化工作空间的代码如下.
\medskip
\medskip

\lstinputlisting[language=Matlab]{task1.m}


\section{避障路径规划}
\subsection{实现思路}
欲控制机械臂避开障碍物从起点到达终点, 只需观察出安全的末端执行器路径, 选取合适的锚点, 利用逆运动学计算出各锚点对应的关节变量, 在这些关节变量之间进行平滑的插值, 最后令机械臂按照插值结果运动即可.\par
本设计中, 障碍物为长方体, 中心为 (100, 0, 50), 大小为 (100, 30, 150), 路径的起点为 (100, 100, 10), 终点为 (100, -100,10). 因此取锚点为: 起点 (100, 100, 10), 起点正下方 (100, 100, -50), 终点正下方 (100, -100, -50), 终点 (100, -100, 10). 基座位置设为 (0, 0, -130). 但是, 在 Matlab 中调用函数 \verb|ikine| 求解逆运动学时出现了无法收敛的错误. 推测这是由于传入的末端执行器朝向无法达到. 可达的朝向是难以获知的, 而事实上, 这里我们只关心末端执行器的位置而并不关心其朝向, 而位置仅由前三个关节变量决定. 因此考虑推导其前三个关节变量的解析解, 后三个关节变量设为 0 即可.\par
给定末端位置 $(x,y,z)$, 下面求解其逆运动学.\par
机械臂前4个关节的齐次变换矩阵计算如下, 其中 $\theta_4$ 设为0.
\[
^0T_1 = 
\begin{pmatrix}
c_1 & -s_1 & 0 & 0\\
s_1 & c_1 & 0 & 0\\
0 & 0 & 1 & 0\\
0 & 0 & 0 & 1
\end{pmatrix}, \quad
^1T_2 = 
\begin{pmatrix}
c_2 & -s_2 & 0 & 0\\
0 & 0 & 1 & 0\\
-s_2 & -c_2 & 0 & 0\\
0 & 0 & 0 & 1
\end{pmatrix},
\]
\[
^2T_3 = 
\begin{pmatrix}
c_3 & -s_3 & 0 & 100\\
s_3 & c_3 & 0 & 0\\
0 & 0 & 1 & 20\\
0 & 0 & 0 & 1
\end{pmatrix}, \quad
^3T_4 = 
\begin{pmatrix}
1 & 0 & 0 & 10\\
0 & 0 & 1 & 100\\
0 & -1 & 0 & 0\\
0 & 0 & 0 & 1
\end{pmatrix}.
\]
进而可得
\[
^0T_2 = ^0T_1 ^1T_2 = 
\begin{pmatrix}
c_1 c_2 & -c_1 s_2 & -s_1 & 0\\
s_1 c_2 & -s_1 s_2 & c_1 & 0\\
-s_2 & -c_2 & 0 & 0\\
0 & 0 & 0 & 1
\end{pmatrix},
\]
\[
^0T_3 = ^0T_2 ^2T_3 = 
\begin{pmatrix}
c_1 c_{23} & -c_1 s_{23} & -s_1 & 100c_1c_2-20s_1\\
s_1 c_{23} & -s_1 s_{23} & c_1 & 100s_1c_2+20c_1\\
-s_{23} & -c_{23} & 0 & -100s_2\\
0 & 0 & 0 & 1
\end{pmatrix},
\]
\[
^0T_4 = ^0T_3 ^3T_4 = 
\begin{pmatrix}
c_1 c_{23} & -c_1 s_{23} & -s_1 & 10c_1c_{23}-100c_1s_{23}+100c_1c_2-20s_1\\
s_1 c_{23} & -s_1 s_{23} & c_1 & 10s_1c_{23}-100s_1s_{23}+100s_1c_2+20c_1\\
-s_{23} & -c_{23} & 0 & -10s_{23}-100c_{23}-100s_2\\
0 & 0 & 0 & 1
\end{pmatrix},
\]
由于后三个关节变量不影响末端执行器位置, 因此末端位置为
\[
\mathbf{p} =
\begin{pmatrix}
10c_1c_{23}-100c_1s_{23}+100c_1c_2-20s_1\\ 10s_1c_{23}-100s_1s_{23}+100s_1c_2+20c_1\\ -10s_{23}-100c_{23}-100s_2
\end{pmatrix}.
\]
于是可以列出方程组
\begin{subequations}
    \label{fkine}
    % \left\{
    \begin{align}
     &x = 10c_1c_{23}-100c_1s_{23}+100c_1c_2-20s_1 \label{x} \\
     &y = 10s_1c_{23}-100s_1s_{23}+100s_1c_2+20c_1 \label{y} \\
     &z = -10s_{23}-100c_{23}-100s_2 \label{z}
    \end{align}
    % \right.
\end{subequations}
设
\begin{equation}
\label{A}
A = 10c_{23}-100s_{23}+100c_2,
\end{equation}
则式 \eqref{x} 和式 \eqref{y} 可写成
\begin{subequations}
    % \left\{
    \begin{align}
     &x = Ac_1 - 20s_1 \label{Ax} \\
     &y = As_1 + 20c_1 \label{Ay}
    \end{align}
    % \right.
\end{subequations}
$\eqref{Ax}^2 + \eqref{Ay}^2$, 得
\begin{equation}
\label{Axy}
A^2 = x^2 + y^2 - 400,
\end{equation}
$\eqref{A}^2 + \eqref{z}^2$, 整理可得
\begin{equation}
    \frac{A^2+z^2}{100} = 201 + 20 (c_3 - 10s_3),
\end{equation}
将 \eqref{Axy} 代入上式, 并设
\begin{equation}
    \label{rho}
    \rho^2 = x^2 + y^2 + z^2,
\end{equation}
整理可得,
\begin{equation}
    \label{preq3}
    c_3 - 10s_3 = \frac{\rho^2 - 20500}{2000},
\end{equation}
设
\begin{equation}
    \label{B}
    B = \frac{\rho^2 - 20500}{2000},
\end{equation}
将其代入式 \eqref{preq3}, 并应用辅助角公式, 可得
\begin{equation}
    \label{q3}
    \sqrt{101}\sin(\theta_3 + \phi_1) = B,
\end{equation}
其中 $\phi_1 = \arctan(\frac{1}{-10}) + \pi$. 由此即可解出 $\theta_3$.\par
将解得的 $\theta_3$ 代回式 \eqref{z}, 整理可得
\begin{equation}
    \label{preq2}
    (B+10)s_2 + (s_3+10c_3)c_2 = \frac{z}{-10}.
\end{equation}
设
\begin{equation}
    \label{C}
    C = B + 10,
\end{equation}
\begin{equation}
    \label{D}
    D = s_3+10c_3.
\end{equation}
注意到, $C = (\rho^2 - 500)/2000$, 而本设计选取的路径上的点显然都满足到基座的距离大于 $\sqrt{500}$, 因此可以保证 $C > 0$. 从而, 对式 \eqref{preq2} 应用辅助角公式, 并将式 \eqref{C} 和式 \eqref{D} 代入, 可得
\begin{equation}
    \label{q2}
    \sqrt{C^2+D^2} \sin(\theta_2 + \phi_2) = \frac{z}{-10},
\end{equation}
其中 $\phi_2 = \arctan(D/C)$. 由此即可解出 $\theta_2$.\par
$\eqref{Ay} \times A - \eqref{Ax} \times 20$, 结合式 \eqref{Axy}, 整理可得
\begin{equation}
    \label{q1}
    s_1 = \frac{yA - 20x}{x^2+y^2}.
\end{equation}
将解得的 $\theta_2,\theta_3$ 代入式 \eqref{A} 即可得到 $A$ 的值. 考虑到 $\theta_1 \in (-\pi/2, \pi/2)$, 由式 \eqref{q1} 立刻得到
\begin{equation}
    \label{solvedq1}
    \theta_1 = \arcsin(\frac{yA - 20x}{x^2+y^2}).
\end{equation}
至此, $\theta_1,\theta_2,\theta_3$ 的解析解已全部可以给出. 需要说明的是, $\theta_2,\theta_3$ 的值分别由式 \eqref{q2} 和式 \eqref{q3} 给出, 但这两个式子各自可能存在两个解. 因此, 在计算出可能的解后, 应将结果代回正运动学公式 \eqref{fkine} 进行验证. 此外, 还需要考虑 $\theta_2,\theta_3$ 的值应在 $(-\pi/2,\pi/2)$ 范围内. 具体的实施细节见下文列出的 \verb|my_ikine| 函数代码.


\subsection{结果}
按照以上思路, 计算出的机械臂从起点到终点的工作空间路径如图 \ref{path} 所示, 各子图展示了机械臂在路径的起点、中点、终点时的姿态. 图中, 两个红色的点代表起点和终点, 橙色的曲线代表路径.\par
机械臂运动过程中, 其各关节角度的变化曲线如图 \ref{angle} 所示. 注意 $\theta_4,\theta_5,\theta_6$ 始终为 0, 因而其曲线重合.\par
\begin{figure}[H]
    \centering
    \subfloat[\label{path-s}起点处]{\includegraphics[width=0.45\linewidth]{task2_path_start.png}}
    \\
    \subfloat[\label{path-m}中点处]{\includegraphics[width=0.45\linewidth]{task2_path_mid.png}}
    \\
    \subfloat[\label{path-e}终点处]{\includegraphics[width=0.45\linewidth]{task2_path_end.png}}
    \caption{\label{path}机械臂从起点到终点的工作空间路径及在运动中的姿态}
\end{figure}
\medskip
\begin{figure}[H]
    \centering
    \includegraphics[width=0.8\linewidth]{task2_curve.png}
    \caption{机械臂各关节角度的变化}
    \label{angle}
\end{figure}



\subsection{代码}
计算逆运动学的代码如下.\par
\medskip
\lstinputlisting[language=Matlab]{my_ikine.m}
\medskip
规划避障路径的代码如下.\par
\medskip
\lstinputlisting[language=Matlab]{task2.m}



\end{document}
